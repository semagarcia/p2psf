\chapter{Conclusiones}
Llegados a este punto es momento de hacer una pausa, respirar hondo y evaluar el trabajo realizado, así como ver todas las piedras que nos hemos ido
encontrando por el camino... a pesar del pedregoso camino, podemos decir satisfactorios y contentos, que el trabajo llegó a buen puerto.\\

Ha sido duro, difícil, requerido de muchas horas, han surgido multitud de dudas, discusiones, formas de poder optimizar mejor el funcionamiento, el
análisis, diseño e implementación, no sólo de toda la aplicación, sino de uno de los elementos clave: el algoritmo de reparto y petición de piezas.\\

Observando todo esto desde una vista de pájaro, podemos, a modo de resumen, destacar:
\begin{itemize}
   \item El gran reto que supone desarrollar una aplicación de tal magnitud en tan poco tiempo.
   \item La gran cantidad de aspectos considerables que abarcan los sistemas distribuídos, y por lo tanto, su desarrollo implica gran complejidad.
   \item Con motivo del punto anterior, existen numerosos términos a los cuales hay que prestarle especial atención, como por ejemplo, control de
         concurrencia, sincronización entre hilos, manejo de fallos u optimización entre otros.
   \item La gran cantidad de clases/objetos e interfaces obtenidos.
   \item La gran cantidad de pasos de mensajes entre todos los objetos que conforman la aplicación.
   \item Y por supuesto, ¡debido a las grandes peculiaridades de CORBA! (tipos de datos, definición de nuevos tipos, paso de objetos con elementos 
         nulos...).
   \item La ingente (e insuficiente) batería de pruebas que requiere, al menos, este tipo de aplicaciones.
   \item Según las opiniones y comentarios de los compañeros de clase y del profesor, ha sido un trabajo formidable.
   \item La satisfacción personal de haber trabajado en grupo y haber obtenido este resultado excelente.\\
\end{itemize}

