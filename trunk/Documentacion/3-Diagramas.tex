\chapter{Diagramas de modelado}
A continuación se exponen los diagramas de paquetes, de clases y uno adicional donde se detallan aspectos concretos de la aplicación.\\

   \section{Diagrama de paquetes}
   En el diagrama de paquetes que se muestra en la figura \ref{diagramaPaquetes} podemos observar que el módulo ``GUI'', que contiene todo el
   código fuente relativo a la interfaz gráfica, sólo interactúa con el módulo cliente, que se considera el núcleo de la aplicación en la parte
   cliente.\\

  \begin{figure} [H] \begin{center}
    \includegraphics[width=0.82\textwidth]{./imagenes/3-DiagPaquetes}
    \caption{Diagrama de paquetes} \label{diagramaPaquetes}
  \end{center} \end{figure}


   \section{Diagrama de clases}
   A continuación se exponen los diagramas de clases de cada uno de los módulos identificados anteriormente.\\
  \begin{figure} [H] \begin{center}
    \includegraphics[width=0.82\textwidth]{./imagenes/3-DiagClases-Coordinador}
    \caption{Diagrama de clases - Coordinador} \label{DCcoord}
  \end{center} \end{figure}

  \begin{figure} [H] \begin{center}
    \includegraphics[width=0.82\textwidth]{./imagenes/3-DiagClases-Cliente}
    \caption{Diagrama de clases - Cliente} \label{DCclient}
  \end{center} \end{figure}

  \begin{figure} [H] \begin{center}
    \includegraphics[width=0.82\textwidth]{./imagenes/3-DiagClases-GUI}
    \caption{Diagrama de clases - GUI} \label{DCGUI}
  \end{center} \end{figure}

   \section{Diagrama de la estructura del cliente}
   Para finalizar este capítulo se expone a continuación el diagrama que muestra cómo está estructurado el cliente, es decir, de qué
   clases hace uso para un mejor entendimiento del funcionamiento de la aplicación cliente.\\
  \begin{figure} [H] \begin{center}
    \includegraphics[width=0.82\textwidth]{./imagenes/3-DiagramaEstructuraUsuario}
    \caption{Cliente - Estructuración interna} \label{DEU}
  \end{center} \end{figure}
   Como se puede ver, el \textit{UsuarioClient} utiliza un objeto \textit{UsuarioServer} que a su vez utiliza un objeto de la clase 
   \textit{UsuarioImpl}. Esta interacción es la que implementa la parte cliente y servidora de cada usuario. \textit{UsuarioClient} es la parte
  cliente, mientras que \textit{UsuarioServer} es la parte servidora que contiene un sirviente \textit{UsuarioImpl}.\\

   Por otro lado, en cuanto a las descargas, la clase \textit{UsuarioClient} puede instanciar cero o más objetos de la clase \textit{Downloader}.
   El objeto Downloader es el encargado de gestionar cada una de las descargas que se estén realizando. A su vez, éste objeto puede instanciar
   tantos objetos \textit{Peticion} como sea necesario. ¿Quién determinará el número de instancias? Dos elementos:
   \begin{itemize}
    \item El número de conexiones máximas: como máximo habrá \textit{numConex} objetos instanciados en un mismo instante.
    \item El número de usuarios a los que pedir partes (peers en primer lugar y seeds en segundo lugar). Si no hay ningún usuario que posea una parte
          que el cliente solicita, no existirá ningún objeto Petición.\\
   \end{itemize}

   Este objeto Petición será el que, en función de la parte que determine el algoritmo del objeto Downloader, pida una porción del fichero, desde un
   inicio hasta un fin, en bloques de un tamaño preasignado, pero configurable\footnote{Este parámetro es delicado, pues un tiempo alto puede provocar,
   cuando la red se encuentre sobrecargada, fallos de tiempo de expiración (timeout) y será necesario el reenvío de los bloques, lo que repercutirá en
   el aprovechamiento de la comunicación}.\\

   Por último, destacar que los objetos Downloader y Petición son hilos, pues extienden de la clase Thread. Así de esta forma podemos crear descargas
   que se realicen de manera concurrente y no sobrecargamos el sistema.

