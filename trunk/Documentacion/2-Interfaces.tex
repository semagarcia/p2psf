\chapter{Interfaces IDL}
A continuación se expone el código correspondiente a la interfaz IDL desarrollada, para posteriormente pasar a su descripción.\\

\begin{verbatim}
/**

* Modulo cliente. Se desarrolla todo el codigo necesario para la comunicación
* entre los usuarios y el almacenamiento del estado de los ficheros localmente.
*/
module cliente {

	// Estructura para almacenar la informacion de los archivos.
	struct infoArchivo {
		string ruta;
		string nombre;
		long long tam;
		long long checksum;
	};

	// Estructura para almacenar las partes de un archivo que el usuario posee o desea descargar.
	struct parteArchivo {
		long long inicio;
		long long fin;
		boolean pedido;
		boolean descargado;
	};

	typedef sequence <parteArchivo> vPartes;

	// Estructura para que los usuarios envíen la información de sus archivos al coordinador.
	struct EstrArchivo {
		infoArchivo info;
		vPartes partes;
	};

	// Definicion de array de EstrArchivos.
	typedef sequence <EstrArchivo> estrArchivos;

	// Secuencia de bytes que los usuarios devolveran como respuesta a las peticiones de archivos.
	typedef sequence <octet> vByte;

	// Sirviente que proporciona partes de un archivo a los usuarios que se la solicitan.
	interface Usuario {
		// Método que devuelve la parte solicitada de un archivo.

		vByte solicitarParte(in string nombre, in long long inicio, in long long fin);
	};
};


/*
* Modulo coordinador. Contiene el codigo necesario para almacenar el estado de los archivos en la red y
* facilita a los clientes la informacion de dichos archivos para posibilitar su descarga de otros usuarios.
*/

module coordinador {
	//Definición de array de int.
	typedef sequence <long> vInt;
	
	// Creado por el coordinador. Mantiene información de los archivos de la red.
	interface Archivo {
		readonly attribute string nombre;
		readonly attribute long long tam;
		readonly attribute long long checksum;

		void insertarSeed(in long id);
		void insertarPeer(in long id, in cliente::vPartes partes);
		void actualizarPartes(in long id, in cliente::vPartes partes);
		void eliminarSeed(in long id);
		void eliminarPeer(in long id);
		vInt getSeeds();
		vInt getPeers();
		cliente::vPartes getPartes(in long id); // Devuelve las partes que posee el usuario con identificador id
	};

	// Sirviente que permite a los usuarios conectarse a la red, desconectarse y recuperar informacion sobre los archivos.
	interface Coordinador{
		long conectar(in cliente::estrArchivos eas, in cliente::Usuario usuario);
		void desconectar(in cliente::estrArchivos eas, in long idUsuario);
		void anyadirArchivos(in cliente::estrArchivos eas, in long idUsuario);
		void eliminarArchivos(in cliente::estrArchivos eas, in long idUsuario);
		Archivo buscar(in string nombre);
		cliente::Usuario getUsuario(in long id);
	};
};
\end{verbatim}


   % Explicación de la interfaz IDL
   \section{Módulo \textit{cliente}}
   Comenzaremos describiendo la interfaz IDL por el módulo cliente, que es el que realiza las peticiones inicialmente al servidor o cuando desea buscar un
   fichero en la red.\\      

      \subsection{Estructura infoArchivo}
      \begin{center}
         \begin{verbatim}
	struct infoArchivo {
		string ruta;
		string nombre;
		long long tam;
		long long checksum;
	};
         \end{verbatim}
      \end{center}
      Esta estructura permite mantener, por cada archivo que se está compartiendo, información relativa al ``sistema de archivos''; es decir, nos
      va a almacenar atributos como el nombre y ruta completa del archivo, longitud y suma de verificación (checksum).\\ 
      De esta forma encapsulamos en esta estructura todo lo relativo al fichero físico en la máquina del cliente.\\

      \subsection{Estructura parteArchivo}
      \begin{center}
         \begin{verbatim}
	struct parteArchivo {
		long long inicio;
		long long fin;
		boolean pedido;
		boolean descargado;
	};
         \end{verbatim}
      \end{center}
      La siguiente estructura que posee el módulo cliente es la que va a permitir almacenar información relativa a las descargas, es decir, aquella
      que nos indica las partes que tiene el cliente, o si éste está lanzando peticiones para solicitar nuevas partes.\\

      Así es fácil mantener, para todos los archivos, las partes que se poseen con el fin de optimizar al máximo la utilización de la red P2P de 
      manera que se pueden ir solicitando partes que ya posean otros usuarios o se puede pausar una descarga (cerrando incluso la aplicación) para
      posteriormente, al volverla a abrir, mantener la consistencia suficiente sobre el archivo de manera que no se pierda información.\\

      \subsection{Estructura EstrArchivo}
      \begin{center}
         \begin{verbatim}
	struct EstrArchivo {
		infoArchivo info;
		vPartes partes;
	};
         \end{verbatim}
      \end{center}      
      Esta estructura, como se puede deducir del fragmento de código, es la ``unión'' de las dos estructuras vistas anteriormente, de manera que
      volvemos a utilizar el concepto encapsulamiento y en un solo objeto mantenemos toda la información necesaria sobre el archivo: nombre físico,
      ruta completa, tamaño, si es un seed o un peer, qué partes posee en caso de estar incompleto... Esta será la información que los usuarios 
      proporcionarán al coordinador.\\

      \subsection{Interfaz Usuario}
      \begin{center}
         \begin{verbatim}
	interface Usuario {
		// Método que devuelve la parte solicitada de un archivo.

		vByte solicitarParte(in string nombre, in long long inicio, in long long fin);
	};
         \end{verbatim}
      \end{center}  
      La interfaz y último elemento que comentaremos sobre el módulo cliente es \textit{Usuario}. Esta interfaz posee un método llamado 
      \textit{solicitarParte}, la cual devuelve un array de bytes. Éste solicita, a partir de un nombre de fichero, el fragmento comprendido entre
      las posiciones \textit{inicio} y \textit{fin}, devolviendo el vector \textit{[inicio, fin]}.\\


   \section{Módulo \textit{coordinador}}
   Ahora nos movemos al otro extremo de la comunicación\footnote{Recordemos que nos estamos refiriendo a una red P2P híbrida o semi-descentralizada.}
   donde se encuentra el servidor de la red.\\

      \subsection{Interfaz Archivo}
      \begin{center}
         \begin{verbatim}
	interface Archivo {
		readonly attribute string nombre;
		readonly attribute long long tam;
		readonly attribute long long checksum;

		void insertarSeed(in long id);
		void insertarPeer(in long id, in cliente::vPartes partes);
		void actualizarPartes(in long id, in cliente::vPartes partes);
		void eliminarSeed(in long id);
		void eliminarPeer(in long id);
		vInt getSeeds();
		vInt getPeers();
		cliente::vPartes getPartes(in long id); // Devuelve las partes que posee el usuario con identificador id
	};
         \end{verbatim}
      \end{center} 
      La interfaz Archivo es la encargada de lo relacionado con las descargas de cada archivo, es decir, de añadir y quitar seeds, de añadir, quitar
      y actualizar peers, así como comprobar los seeds y peers que hay de un determinado archivo.\\

      \subsection{Interfaz Coordinador}
      \begin{center}
         \begin{verbatim}
	interface Coordinador{
		long conectar(in cliente::estrArchivos eas, in cliente::Usuario usuario);
		void desconectar(in cliente::estrArchivos eas, in long idUsuario);
		void anyadirArchivos(in cliente::estrArchivos eas, in long idUsuario);
		void eliminarArchivos(in cliente::estrArchivos eas, in long idUsuario);
		Archivo buscar(in string nombre);
		cliente::Usuario getUsuario(in long id);
	};
         \end{verbatim}
      \end{center} 
      La última interfaz que abordamos es la del Coordinador, a través de la cual se podrán realizar las siguientes funciones:
      \begin{itemize}
         \item Registrar a un nuevo usuario en la red, asignándole un nuevo identificador interno. Este registro se lleva a cabo mediante el par de
               valores \textit{(identificador, referenciaCORBA)}. Lo mismo con el proceso inverso para la desconexión.
         \item Compartir y añadir nuevos ficheros a la red P2PSF, es decir, registrar y publicar que hay más ficheros compartidos. Análogo cuando un
               usuario se desconecta y deja de compartir sus recursos.
         \item Realizar búsquedas en la red P2PSF a petición de un usuario para descargar algún archivo.
         \item Obtener la referencia CORBA (interfaz Usuario) de un cliente de la red para solicitarle una o más partes de un archivo deseado.
      \end{itemize}



